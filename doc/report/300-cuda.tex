\section{CUDA}
\label{sec:cuda}

%\todo[inline]{Explain CUDA}

Compute Unified Device Architecture, or CUDA\tm is a parallel computing platform developed by NVidia\tr that enabled massive performance increases for highly data-parallel applications, by providing a programming model more suited to graphics processing units.

Contrary to a CPU, which focus on executing a small amount of threads quickly, by emplying methods like branch-prediction and superscalarity, a GPU follows a different filosophy, executing much more threads slowly. Even though each individual thread runs more slowly, that latency is hidden by the fact that many more threads are executing concurrenlty, and the hardware level scheduler can switch the context with only a few clock cycles.

Of course, since the programming model is completely different, a CPU thread isn't comparable in any way to a CUDA thread. For example, while a simple algorithm might consist in looping through a collection of elements, applying an operation to each one, in CUDA a more suitable model would be to launch one thread for each element, and let the hardware scheduler manage them.

\subsection{Programming Model}
\label{sec:cuda:model}

A CUDA device (GPU) tipically consists on a group of Streaming Multiprocessors (SM's), each one containing up to 64 CUDA Cores \footnote{This refers to the CUDA architecture up until Fermi. The new Kepler architecture features 192 CUDA Cores in each new extended Streaming Multiprocessor (SMX)}
This Cores are the the elementary processing unit of a GPU. Each one is capable of one simple sequential operation at a time (arithmetic operations for example). More complex operations, like trigonometric functions, and square roots are executed on four Special Function Units, which process them at a rate of one instruction per clock-cycle.

Each SM can only issue one instruction each clock cycle, which means that all CUDA Cores of a given SM must execute the same instruction (altough the indexing of each CUDA Thread allows them to operate on different data). Thus, threads are grouped into warps, 32 Threads each, which are executed at the same time on an SM. Two schedulers per SM allow near peak performance by selecting two instructions from two different warps to be issued every clock cycle, increasing concurrency.

Note that this greatly limits concurrency when introducing branches. If threads within the same warp diverge in a branch, the entire warp will be required to wait as much as it would have if all threads executed both branch statements. This can be optimized by the programmer, if the branch pattern can be recognized to group non divergent threads in the same warps

Besides warps, threads are also grouped into blocks, which in turn are grouped int a grid. A block is the main organizational unit of threads. A single block can contain up to 1024 threads, and is assigned to a SM at the beggining of a kernel. The SM will later organize those threads into warps.

A grid is nothing more than a collection of blocks, organized in a bidimensional space, where each block will be assigned to a specific SM to execute the kernel function.

\subsubsection{Synchronization}
\label{sec:cuda:model:sync}

Having thousands of threads executing concurrently raises the issues of synchronization already familar from other parallel models. Sometimes a barrier is required to allow other threads to compute results that the current thread depends on.

In CUDA, synchronization is implicitl within a single warp, since all threads of a warp will execute the same instruction. On a higher level, the CUDA primitive \texttt{\_\_syncthreads()} is available to create a barrier for all threads within a block, allowing the entire block to reach the same point.

As for grid level synchronization, it is not possible within the GPU, since each SM will run their blocks independently, but it can be achieved by using different kernel calls, implicitly creating a synchronization point between kernels.
\subsection{Profiling CUDA}
\label{sec:cuda:prof}

\todo[inline]{Explain (no need for much detail) what can be profiled in a GPU,, how it works, what info can be extracted from the results, and how it differs from CPU profiling. Maybe add subsubsections?}
\todo[inline]{Somewhere in here, talk about built-in profilers, like CUPTI}

Since CUDA 4.0, profiling is done mostly via the CUDA Profiling Tools Interface (CUPTI). This library provides four main API's: Activity, Callback, Event and Metric. These are described in the next sections.

\subsubsection{Activity API}
\label{sec:cuda:prof:activity}

The Activity API allows asynchronous recording of all CUDA activity, both on CPU and GPU. Each Activity type as an associated ata structures that is used to provide details on each individual activity.
\subsubsection{Callback API}
\label{sec:cuda:prof:callback}

The CUPTI Callback API allows the programmer to register callbacks in it's own code, which will be invoked upon calling a CUDA runtime or driver function.

Not only is this useful when profiling, it might also be a good tool to automate a task that the programmer wants to execute each time a specific event occurs (for example, create a custom log of all memory transactions between the CPU and the GPU).

\subsubsection{Event API}
\label{sec:cuda:prof:event}

This API provides accesses to counters on a CUDA-enabled device. Much like PAPI, although with some differences due to architectural aspects, this counters provide access to hardware level counters to measure specific events on the GPU.

The workflow of this API is straighforward, by simply defining a group of events to measured prior to the kernel invocation, and retrieving the results at the end.

Some care needs to be taken with this, as event registration is asynchronous, meaning it is the programmer's responsibility the relevant kernel code only starts to run after the registration, and not thread-safe, meaning that, if other threads in the application are also using the GPU, chances are they are concurrently executing kernels, which will interfere with the measurements.

The set of events that can be measured is dependant on the compute capability of the device used. Prior to compute capability 2.0, the event set was quite limited, allowing only measurements of cache hits, misses, number of instructions (optionally of a given type), divergent branches, and counters related to memory accesses.

There are also 8 generic purpose triggers that can be used by the programmer to measure specific programmer-defined events in the code.

As for Compute Capability 2.0 and greater, most of the events from prior versions are still compatible, but a lot more are added. These events are split over for domains:

\begin{description}
	\item[\texttt{domain\_a}] For events related with cache misses/hits, conflicts. Also relates to texture cache;

	\item[\texttt{domain\_b}] Various read/write requests for each memory hierarchy level (cache, Texture, DRAM);

	\item[\texttt{domain\_c}] For global load/store operation counters. Multiple counters are provided for the different amount of bytes requested.

	\item[\texttt{domain\_d}] Instruction related counters, like number of branches, divergences, active warps/cycles, number of instructions and memory requests. 8 general purpose counters are also included in this domain

\end{description}

In addition to the event set, other differences distinguish CUDA devices with compute capability below 2.0. Perhaps the most important of them is the fact that on those devises, all events are counted for only one SM. Starting on compute capability 2.0, events from \texttt{domain\_d} are counted for multiple (but no all) SM's. Therefore, to get the most consisted results, it is best to use a grid that has a number of blocks multiple of the total number of SM's in the device used.
\subsubsection{Metric API}
\label{sec:cuda:prof:metric}

CUPTI's Metric API calculates metrics for each kernel, based on one or more event values. These metrics might be useful to get a higher level overview of the GPU usage and efficiency before having to look at the more raw information provided by the Event API.

Like the Event API, it evolved in version 2.0. Initial versions only provided a small set of metrics:

\begin{description}
	\item[Branch efficiency] Shows the overall ratio of non-divergent branches to total branches, effectively indicating whether or not branching operations are greatly hindering performance.

	\item[Load/Store efficiency] Can be useful to assert if memory accesses are sufficiently coherent to take advantage of memory coalescence.

	\item[Memory Throughput] For metrics indicating effective throughput of memory load/store operations
\end{description}

Compute Capability 2.0, in addition to the previous metrics, provides additional ones:

\begin{description}
	\item[SM Efficiency] Ratio of time at least one warp was active on a multiprocessor to total time

	\item[Occupancy] Ratio of average active warps per active cycle to the maximum amount of warps supported

	\item[IPC] Instructions per Cycle

	\item[Replay overhead] The performance loss due to memory replays

	\item[Cache Hit Rate] Hit rate of L1 cache for both global/local loads and stores

\end{description}

\section{Command Line Profiler}
\label{sec:330}

Besides the CUPTI APIs, NVidia also created the Command Line Profiler, providing simpler profiling support for CUDA applications, and more recently, OpenCL as well.

This profiler works in a unobtrusive way, allowing previously compiled applications to be targeted with some degree of profiling. This also requires almost no effort from the programmer, who only has to specify some environment variables, indicating the CUDA driver that he wants the profiling to be done.

Given the simplicity, which is good for when only a rough idea about the profiling results is required, there are obvious drawbacks when compared to more powerful and feature heavy profilers like CUPTI.

First of all, since CUPTI is nothing more than a API to allow communication with the CUDA driver and retrieve hardware level information, it can certainly be more tailored to specific needs, providing only the desired results, and measuring nothing more than necessary, making the output also more efficient and user friendly.

With the command line tool, the degree of customization is extremely low. One might want to measure two kernels of the same application using different counters. With CUPTI this simply means two event sets that are initialized at different kernel launches. But with the command profiler, either both kernels are measured for both event sets, or two different executions are required, with each one receiving a different counter list.

Aside from that, the same counters from CUPTI are available. They are specified using an environment variable linking to a file that contains one counter name per line. Some counters are incompatible, meaning they cannot be measured in the same run (this can also be an issue in CUPTI, although its solvable by measuring two event sets separately).

The output can either be in the raw profiler format, or in CSV format. But event though a CSV file can easily be parsed into a spreadsheet, the output can still be badly organized. Each line in the file corresponds to a single kernel call, with the columns specifying each counter value for that launch. If counters from different domains are requested at the sametime, the output will not be as much organized as it would be desirable.

In conclusion, this profiler sees it's best use when only a small amount of information is required, with little to no data treatment, and where CUPTI would require unnecessary effort to program.
\subsection{Visual Profiler}
\label{sec:340}

\todonaps{Talk about this thing}