\documentclass[abstract=on,10pt,twocolumn]{scrartcl}

\usepackage{ucs}
\usepackage[utf8x]{inputenc}
\usepackage[T1]{fontenc}
\usepackage[english]{babel}
\usepackage{datetime}

\usepackage[paper=a4paper,top=2cm,left=1.5cm,right=1.5cm,bottom=2cm,foot=1cm]{geometry}

\usepackage{relsize}%	relative font sizes

\usepackage[retainorgcmds]{IEEEtrantools}%	IEEEeqnarray
\setlength{\IEEEnormaljot}{4\IEEEnormaljot}

\usepackage{indentfirst}
\usepackage{hyperref}
\usepackage{cleveref}
\usepackage{color}

\usepackage{todonotes}
\usepackage{hyperref}
\usepackage{cleveref}
\usepackage[cmex10]{amsmath}
\usepackage{epstopdf}
\usepackage{graphicx}


%%%%%%%%%%%%%%%%
%  title page  %
%%%%%%%%%%%%%%%%
\titlehead{University of Minho \hfill Master's Degree in Informatics Engineering\\Department of Informatics \hfill Computer Graphics - Integrated Project}

\title{Performance API for CUDA}
\subtitle{A critical analysis}

\author{
	\\Ana Catarina Macedo\\
		\texttt{\smaller a54XXX@alunos.uminho.pt}
	\and
    \\Miguel Palhas\\
     	\texttt{\smaller pg19808@alunos.uminho.pt}
	\and
	\\Pedro Costa\\
		\texttt{\smaller pg19830@alunos.uminho.pt}
}

\date{Braga, \docdate}

%\subject{PAPI CUDA}

%%%%%%%%%%%
%  Hacks  %
%%%%%%%%%%%

%	Paragraph (title) with linebreak
\newcommand{\paragraphh}[1]{\paragraph{#1\hfill}\hfill

}

%	Add "Appendix" to the appendices titles, but not to the references
\usepackage{ifthen}
\newcommand*{\appendixmore}{%
  \renewcommand*{\othersectionlevelsformat}[1]{%
    \ifthenelse{\equal{##1}{section}}{\appendixname~}{}%
    \csname the##1\endcsname\autodot\enskip}
  \renewcommand*{\sectionmarkformat}{%
    \appendixname~\thesection\autodot\enskip}
}

\newdateformat{mmmyyyydate}{\monthname[\THEMONTH] \THEYEAR}
\newcommand{\docdate}{\mmmyyyydate\today}

\newcommand{\tm}{\textsuperscript{\texttrademark}}
\newcommand{\tr}{\textsuperscript{\textregistered1342}}



\begin{document}
	\maketitle

	\todo[inline]{corrigir o num de aluno da guitar}
	\begin{abstract}
PAPI CUDA presents the CUDA component added to the PAPI library to extend its functionality for massively parallel programs. Many tools already exist in the \nvidia SDK for the purpose of profiling. PAPI CUDA stands as a wrapper for the low-level library CUPTI, which is supposed to aid developers in the deep profiling process.

Tests show that the PAPI library does not provide further functionality over CUPTI and is not easier to use. Compared to the higher level tools, its analysis also stood defective and too complex.
\end{abstract}

	\section{Introduction}

\todo[inline]{Intro}
	\section{Performance API}
\label{sec:papi}

%\todo[inline]{short papi description (no cuda yet)}

Performance API \cite{papi} (commonly referred to as PAPI) is a library developed by the Innovative Computing  Laboratory from the University of Tennessee that allows programs to access counters and measurement instructions that operate at the hardware level.

Most modern processors provide a set of built-in counters that can be programatically used to keep track of information that may be helpful to profile the program. PAPI allows the programmer to use these counters, using the library's functions, and measure a set of events within a block of code.

The events that can be measured by the library are varied. The most common counters measure events like clock cycles, number of total instructions (from which some can measured separately, like additions, multiplications, divisions, integer or floating-point operations), number of hits and misses in each cache level, branch instructions, or stall cycles.

The actual set of counters that can be measured is architecture-specific. Because of this, PAPI provides tools to show what counters can be measured on a specific machine, and provides a more intuitive alias for each counter, instead of the hexadecimal code provided by the manufacturer.

PAPI also attempts to deal with some of the limitations of hardware counters, like its limited amount. For instance, an \intel \xeon X5650 at 2.67 GHz has only 7 real hardware counters.
For this, PAPI provides multiplexing, where a set of counters issued by the programmer share a time slice of the hardware counter, and the final result is estimated based on the total results, and the percentage of time that each counter consumed. Of course, this does not allow counter results to be nearly as accurate as they would be if measured on separate runs, but given that the measured code block is large enough to minimize multiplexing errors, it should provide a good enough alternative for programs where the execution time hampers multiple executions to be made.

	\section{CUDA}
\label{sec:cuda}

%\todo[inline]{Explain CUDA}

Compute Unified Device Architecture, or CUDA\tm is a parallel computing platform developed by NVidia\tr that enabled massive performance increases for highly data-parallel applications, by providing a programming model more suited to graphics processing units.

Contrary to a CPU, which focus on executing a small amount of threads quickly, by emplying methods like branch-prediction and superscalarity, a GPU follows a different filosophy, executing much more threads slowly. Even though each individual thread runs more slowly, that latency is hidden by the fact that many more threads are executing concurrenlty, and the hardware level scheduler can switch the context with only a few clock cycles.

Of course, since the programming model is completely different, a CPU thread isn't comparable in any way to a CUDA thread. For example, while a simple algorithm might consist in looping through a collection of elements, applying an operation to each one, in CUDA a more suitable model would be to launch one thread for each element, and let the hardware scheduler manage them.

\subsection{Programming Model}
\label{sec:cuda:model}

A CUDA device (GPU) typically consists on a group of Streaming Multiprocessors (SM's), each one containing up to 64 CUDA Cores \footnote{This refers to the CUDA architecture up until Fermi. The new Kepler architecture features 192 CUDA Cores in each new extended Streaming Multiprocessor (SMX)}.
These cores are the the elementary processing unit of a GPU. Each one is capable of one simple sequential operation at a time (e.g., arithmetic operations). More complex operations, like trigonometric functions, and square roots are executed on four Special Function Units, which process them at a rate of one instruction per clock-cycle.

Each SM can only issue one instruction each clock cycle, which means that all CUDA Cores of a given SM must execute the same instruction (although the indexing of each CUDA Thread allows them to operate on different data). Thus, threads are grouped into warps, 32 threads each, which are executed at the same time on an SM. Two schedulers per SM allow near peak performance by selecting two instructions from two different warps to be issued every clock cycle, increasing concurrency.

Note that this greatly limits concurrency when introducing branches. If threads within the same warp diverge in a branch, the entire warp will be required to wait as much as it would have if all threads executed both branch statements. This can be optimized by the programmer, if the branch pattern can be recognized to group non divergent threads in the same warps.

Besides warps, threads are also grouped into blocks, which in turn are grouped in a grid. A block is the main organizational unit of threads. A single block can contain up to 1024 threads (depends on the compute capability of the hardware), and is assigned to a SM at the beginning of a kernel. The SM will later organize those threads into warps.

A grid is nothing more than a collection of blocks, organized in a bidimensional space, where each block will be assigned to a specific SM to execute the kernel function.

\subsubsection{Synchronization}
\label{sec:cuda:model:sync}

Having thousands of threads executing concurrently raises the issues of synchronization already familiar from other parallel models. Sometimes a barrier is required to allow other threads to compute results that the current thread depends on.

In CUDA, synchronization is implicit within a single warp, since all threads of a warp will execute the same instruction. On a higher level, the CUDA primitive \texttt{\_\_syncthreads()} is available to create a barrier for all threads within a block, allowing the entire block to reach the same point.

As for grid level synchronization, it is not possible within the GPU, since each SM will run their blocks independently, but it can be achieved by using different kernel calls, implicitly creating a synchronization point between kernels.


\input{report/320-prof}
	\section{PAPI CUDA}

\begin{frame}
	\frametitle{PAPI CUDA}
	\begin{itemize}
		\item Uses CUPTI;
		\begin{itemize}
			\item Supposed to be a wrapper;
		\end{itemize}
		\vfill
		\item [+] Approach to GPU counters is similar to CPU;
		\vfill
		\item [-] Requires the same ``garbage'' code;
		\vfill
		\item Works best when used with CUPTI Callbacks;
		\begin{itemize}
			\item It should replace them;
			\item Tests performed only obtained the measurements for the global kernel;
			\begin{itemize}
				\item Subkernels were ignored;
				\item Highly verbose preparation does not allow to discard programmer error;
			\end{itemize}
		\end{itemize}
	\end{itemize}
\end{frame}

	\section{Test Case}
\label{sec:test}

\todo[inline]{Explain what the test case is, the tests performed, and the methodology (no need for appendices i think, unless this becomes too large)}
	\section{Results}
\label{sec:results}

\todo[inline]{Results from profiling the test case. Not too extensive, since the interest here is in the analysis of the profiler, and not the actual results}

\subsection{Common Limitations}
\label{sec:610}
	\section{Analysis}
\label{sec:anal}% i know, right?

\todo[inline]{The real juice. Talk shit about stuff. A LOT}
\todo[inline]{Can also serve as a conclusion}



	\nocite*
	\bibliographystyle{cell}
	\bibliography{references}
	%\todo[inline]{Corrigir o aspecto das referencias}

\end{document}